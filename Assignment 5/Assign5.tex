\documentclass{beamer}

\usepackage{enumitem}
\usepackage{tfrupee}
\usepackage{amsmath}
\usepackage{amssymb}
\usepackage{graphicx}
\usepackage{txfonts}

\def\inputGnumericTable{}

\usepackage[latin1]{inputenc}                                 
\usepackage{color}                                            
\usepackage{array}                                            
\usepackage{longtable}                                        
\usepackage{calc}                                             
\usepackage{multirow}                                         
\usepackage{hhline}                                           
\usepackage{ifthen}
\usepackage{caption} 
\captionsetup[table]{skip=3pt}  
\providecommand{\pr}[1]{\ensuremath{\Pr\left(#1\right)}}
\providecommand{\cbrak}[1]{\ensuremath{\left\{#1\right\}}}
\renewcommand{\thefigure}{\arabic{table}}
\renewcommand{\thetable}{\arabic{table}}   
\providecommand{\brak}[1]{\ensuremath{\left(#1\right)}}
\newcommand*{\permcomb}[4][0mu]{{{}^{#3}\mkern#1#2_{#4}}}
\newcommand*{\perm}[1][-3mu]{\permcomb[#1]{P}}
\newcommand*{\comb}[1][-1mu]{\permcomb[#1]{C}}

% Theme choice:
\usetheme{CambridgeUS}

% Title page details: 
\title{Assignment 5} 
\author{Santoshi Gayatri Mavuru}
\date{\today}
\logo{\large \LaTeX{}}


\begin{document}

% Title page frame
\begin{frame}
    \titlepage 
\end{frame}

% Remove logo from the next slides
\logo{}


% Outline frame
\begin{frame}{Outline}
    \tableofcontents
\end{frame}

\section{Question}
    \begin{frame}{Question}
    Consider the following three events: 
    \begin{enumerate}
    \item[(i)] At least 1 six is obtained when six dice are rolled.\\
    \item[(ii)] At least 2 sixes are obtained when 12 dice are rolled.\\
    \item[(iii)] At least 3 sixes are obtained when 18 dice are rolled. \\
    \end{enumerate}
    Which of these events is more likely?
    \end{frame}
    
\section{Solution}
\begin{frame}{Solution}
Let X be a random variable representing our required outcomes.
\begin{table}[ht!]
\centering
\input{table/t.tex}
\caption{}
\label{table:TABLE}
\end{table}
\end{frame}


\begin{frame}{Solution}
\begin{align} 
 {\pr{X=0} &= 1 - \text{(No dice shows six)}}\\
&  { = 1 - \brak{\frac{5}{6}}^{6}}\\[6pt]
& {= 0.665}\\
{\pr{X=1} &= 1 - \text{(No dice shows six + One die shows six)}}\\
    & {= 1 - \brak{\brak{\frac{5}{6}}^{12} + \comb{12}{1}\times  \frac{1}{6}\brak{\frac{5}{6}}^{11}}}\\[6pt]
    & {= 0.61866}\\[6pt]
\end{align}
\end{frame}

\begin{frame}{Solution}
\begin{align}
{\pr{X=2} &= 1 -\text{(No dice shows 6 + One die shows 6 + Two dice show 6)}\\
& { = 1 - \brak{\brak{\frac{5}{6}}^{18} + \comb{18}{1}\times \brak{\frac{1}{6}} \brak{\frac{5}{6}}^{17} + \comb{18}{2} \times \brak{\frac{1}{6}}^2\brak{\frac{5}{6}}^{16}}}}\\[6pt]
& { = 0.5973}
\end{align}
Hence obtaining one six when 6 dice are rolled is more likely to occur than the other cases.
\end{frame}
\end{document}