\documentclass[journal,12pt,twocolumn]{IEEEtran}
\pagestyle{empty}
\usepackage{setspace}
\usepackage{amssymb}
\usepackage{amsthm}
\usepackage{amsmath}
\usepackage{mathrsfs}
\usepackage{enumitem}
\usepackage{mathtools}
\usepackage{longtable}
\usepackage[breaklinks=true]{hyperref}

\usepackage{listings}
    \usepackage{color}                                            %%
    \usepackage{array}                                            %%
    \usepackage{longtable}                                        %%
    \usepackage{calc}                                             %%
    \usepackage{multirow}                                         %%
    \usepackage{hhline}                                           %%
    \usepackage{ifthen}                                           %%
    \usepackage{lscape}     
    \usepackage{amsmath}
       
\lstset{
%language=C,
frame=single, 
breaklines=true,
columns=fullflexible
}
\def\inputGnumericTable{}

\bibliographystyle{IEEEtran}
\providecommand{\pr}[1]{\ensuremath{\Pr\left(#1\right)}}
\providecommand{\brak}[1]{\ensuremath{\left(#1\right)}}

\newcommand{\question}{\noindent \textbf{Question: }}
\newcommand{\solution}{\noindent \textbf{Solution: }}
\newcommand*{\permcomb}[4][0mu]{{{}^{#3}\mkern#1#2_{#4}}}
\newcommand*{\perm}[1][-3mu]{\permcomb[#1]{P}}

\title{AI1110 Assignment 4}
\author{Santoshi Gayatri (CS21BTECH11036)}
% make the title area

\begin{document}
\maketitle

\textbf{Question:} Two numbers are selected at random (without replacement) from the first six positive integers. Let X denote the larger of the two numbers obtained. Find E(X). \\
                            
\textbf{Solution:}                           
Consider first n positive integers. \\
Number of ways to select two random numbers(without replacement) from the first n positive positive integers = 
$ \perm{n}{2} = n(n-1)$. \\
Let X represent the larger of the two numbers obtained. Therefore, X can take the values of $$2,3,4,5\ldots, n.$$
i.e. \pr{X=i} where i $\in (2,n)$\\[8pt]
$ \therefore  \pr{X=i} = \displaystyle\frac{2(i-1)}{n(n-1)} $

\begin{align}
 E(X) &=\sum_{i = 2}^{i = n}X_iP(X_i) \\
 & = \sum_{i = 2}^{i = n} i \times \frac{2(i-1)}{n(n-1)}\\
 & = \frac{2}{n(n-1)} \sum_{i = 1}^{i = n} i^2 - \frac{2}{n(n-1)}\sum_{i = 1}^{i = n} i
\end{align}
\begin{align}
	\begin{split}
		&= \frac{2}{n(n-1)} \times \frac{n(n+1)(2n+1)}{6}
 \\ 
 	&	-\frac{2}{n(n-1)} \times \frac{n(n+1)}{2}
	\end{split}
\end{align}	
	
\begin{align}
 & = \frac{2(n+1)}{3}
\end{align}
For n = 6, substituting value of 6 in (5), we get\\
$$\boxed{E(X) = \frac{14}{3} }$$



\end{document}