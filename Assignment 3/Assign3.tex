\documentclass[journal,12pt,twocolumn]{IEEEtran}

\usepackage{setspace}
\usepackage{amssymb}
\usepackage{amsthm}
\usepackage{mathrsfs}
\usepackage{enumitem}
\usepackage{mathtools}
\usepackage{longtable}
\usepackage[breaklinks=true]{hyperref}

\usepackage{listings}
    \usepackage{color}                                            %%
    \usepackage{array}                                            %%
    \usepackage{longtable}                                        %%
    \usepackage{calc}                                             %%
    \usepackage{multirow}                                         %%
    \usepackage{hhline}                                           %%
    \usepackage{ifthen}                                           %%
    \usepackage{lscape}     
    \usepackage{amsmath}
       
\lstset{
%language=C,
frame=single, 
breaklines=true,
columns=fullflexible
}
\def\inputGnumericTable{}

\bibliographystyle{IEEEtran}
\providecommand{\pr}[1]{\ensuremath{\Pr\left(#1\right)}}
\providecommand{\brak}[1]{\ensuremath{\left(#1\right)}}

\newcommand{\question}{\noindent \textbf{Question: }}
\newcommand{\solution}{\noindent \textbf{Solution: }}


\title{AI1110 Assignment III (ICSE Class 12 2018)}
\author{Santoshi Gayatri (CS21BTECH11036)}
% make the title area

\begin{document}
\maketitle

\textbf{Question:} One card is drawn from a well shuffled deck of 52 cards. If each outcome is equally likely, calculate the probability that the card will be
\begin{enumerate}
    \item[(i)]  a diamond
    \item[(ii)] not an ace
    \item[(iii)] a black card (i.e., a club or, a spade)
    \item[(iv)] not a diamond
    \item[(v)] not a black card
\end{enumerate}

\textbf{Solution:}
Let X = {0,1,2,3,4} be a random variable representing outcomes of cards drawn from a deck. 

\begin{table}[ht!]
\centering
\input{tables/table_assign3.tex}
\caption{}
\label{table:table1}
\end{table}

\begin{enumerate}
\item[(i)]{Number of diamonds in a deck are 13.\\
\begin{align} 
\pr{X=0} = \frac{13}{52} = \frac{1}{4}
\end{align} }
\item[(ii)]{ Number of cards which are not ace = 52 - 4 = 48.\\
\begin{align}
\pr{X=1} = \frac{48}{52} =\frac{12}{13}
\end{align}}
\item[(iii)] { Number of black cards(i.e. club or spade) in a deck are 26.\\
\begin{align}
\pr{X=2} = \frac{26}{52} = \frac{1}{2}
\end{align} }
\item[(iv)]{Number of cards which are not diamond = 52 - 13 = 39.\\
\begin{align}
\pr{X=3} = \frac{39}{52} = \frac{3}{4}
\end{align}}
\item[(v)]{Number of  cards that are not black = 52-26 = 26.\\
\begin{align}
\pr{X=4} = \frac{26}{52} = \frac{1}{2}
\end{align}}

\end{enumerate}
\end{document}

